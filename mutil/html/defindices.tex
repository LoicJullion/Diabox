
% This LaTeX was auto-generated from MATLAB code.
% To make changes, update the MATLAB code and republish this document.

\documentclass{article}
\usepackage{graphicx}
\usepackage{color}

\sloppy
\definecolor{lightgray}{gray}{0.5}
\setlength{\parindent}{0pt}

\begin{document}

    
    
\subsection*{Contents}

\begin{itemize}
\setlength{\itemsep}{-1ex}
   \item 1) create indices for unknowns
   \item 2) Create indices for property fluxes \%
   \item 3) Mean area and properties for each isopycnal level
\end{itemize}
\begin{verbatim}
function [indv,indw,indF,indE,indb,meanprop,level_area] = defindices(A,nAbase,nA,nboxes,nproperties,nlayers,nsections)
\end{verbatim}
\begin{verbatim}
%==========================================================================
% defindices.m
%
% Identify the components of A, x and b in DIABOX format, using the data
% structure pointers (see the original DOBOX manual, pg. 24).
%   We define:
%   indv = (column) indices for ref. velocity
%   indw = (column) indices for w.
%   indF = Indices for air-sea forcing corrections (tacked on after w* terms)
%   indE = Indices for Ekman transport corrections
%
%   indb = (row) indices for property fluxes.
%
%   meanprop(i,j,k) is the mean value of property j at level i, in box k
%
% Example:
% x(indv(i)) reference velocity at stn. pair i
% x(indw(i,j,k)) equivalent diapycnal velocity at level i of property j, box k.
% b(indb(i,j,k)) layer i flux of property j, box k.(i=nlayers+1 for vertical integral)
%
% AUTHORS: Rick Lumpkin.
%
% MODIFIED: Loic Jullion (Transformed in a function and commented).
%
%-----------------
% Licence:
% This file is licensed under the Creative Commons Attribution-Share
% Alike 4.0 International license.
%
%     You are free:
%
%         to share ? to copy, distribute and transmit the work
%         to remix ? to adapt the work
%
%     Under the following conditions:
%
%         attribution ? You must attribute the work in the manner specified
%                       by the author or licensor (but not in any way that
%                       suggests that they endorse you or your use of the
%                       work).
%         share alike ? If you alter, transform, or build upon this work,
%                       you may distribute the resulting work only under
%                       the same or similar license to this one.
%-----------------
%
%==========================================================================
\end{verbatim}


\subsection*{1) create indices for unknowns}

\begin{verbatim}
% (column) indices for ref. velocity
indv=(1:nAbase)';
% (column) indices for w.
for k=1:nboxes;
  for j=1:nproperties;
      jj = (k-1)*nproperties+j;
      indw(:,j,k ) =( nAbase+1+(jj-1)*(nlayers-1):nAbase+jj*(nlayers-1) )';
  end;
end;

% indices for air-sea forcing corrections (tacked on after w* terms):
% corrections are for heat and freshwater flux only; corresponding density
% flux corrections are directly calculated from F*,h and F*,fw,
% x(indF(:,1,:)) : correction to H flux
% x(indF(:,2,:)) : correction to FW flux
% NOTE: these terms are defined on the glevels grid, not on glayers.
indF=NaN*ones(nlayers-1,2,nboxes);
ct=1;
for k=1:nboxes;
  for j=1:2;
    for i=1:nlayers-1;
      indF(i,j,k)=max(max(max(indw)))+ct;
      ct=ct+1;
    end;
  end;
end;

% indices for Ekman transport corrections (one multiplicative factor for
% the net Ekman transport across each section, applied to all layers and
% all properties)
indE=NaN*ones(nsections,1);
ct=1;
for k=1:nsections;
  indE(k)=max(max(max(indF)))+ct;
  ct=ct+1;
end;

% done creating indices for unknowns.
\end{verbatim}

        \color{lightgray} \begin{verbatim}Error using defindices (line 51)
Not enough input arguments.
\end{verbatim} \color{black}
    

\subsection*{2) Create indices for property fluxes \%}

\begin{verbatim}
% (row) indices for property fluxes;
% indb(1:nlayers,j,k) are the layer fluxes of property j in box k;
% indb(nlayers+1,j,k) is the vertically-integrated flux of property j in box k.
indb = NaN*ones(nlayers+1,nproperties,nboxes);
for k=1:nboxes;
  for j=1:nproperties;
      jj=(k-1)*nproperties+j;
      indb(:,j,k)=( 1+(jj-1)*(nlayers+1): jj*(nlayers+1) )';
  end;
end;

% done creating property flux indices
\end{verbatim}


\subsection*{3) Mean area and properties for each isopycnal level}


\begin{verbatim}if diapycnal velocities are included Extract mean level areas and
properties from A. meanprop(i,j,k) is the mean value of property j
at level i, in box k.  For j=1, this is the level's area.
(Note: this could alternatively be done by loading
the values from the mprop_atl*.mat files.)\end{verbatim}
    \begin{verbatim}
if (nA>nAbase)

  meanprop=NaN*ones(nlayers-1,nproperties,nboxes);
  for k=1:nboxes;
    for j=1:nproperties;
      meanprop(:,j,k)=diag(A(indb(1:nlayers,j,k),indw(:,j,k)));
    end;
  end;

  % define level_area, the area of the layers
  level_area=meanprop(:,1,:);
  level_area=reshape(level_area,[nlayers-1,nboxes]);
  q=find(level_area==0);
  level_area(q)=NaN;

  % divide by level area to get mean property
  % values (ones for volume, salinity anomaly in
  % psu, temperature in degC., etc.)
  for k=1:nboxes
    for j=1:nproperties
      meanprop(:,j,k)=meanprop(:,j,k)./level_area(:,k);
    end;
  end;

  q=find(meanprop==0);
  meanprop(q)=NaN;

end;
return
\end{verbatim}



\end{document}
    
